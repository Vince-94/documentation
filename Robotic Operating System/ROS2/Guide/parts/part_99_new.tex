\part{ROS2 Graph}

  The ROS graph is a network of ROS2 elements processing data 
  together at one time.


  \section*{Nodes}
    Each node in ROS should be responsible for a single task (single, modular purpose).
    There are many ways by which nodes can exchange data each 
    others: topics, services, actions, or parameters.


  \section*{Topics}
    Topics allows to send continuous streams of data between
    two or more topics:
    \begin{itemize}
      \item publisher: if data is sent (publish to a topic)
      \item subscriber: if data is received (subscribe to a topic)
    \end{itemize}


  \section*{Services}
    Services are based on call-and-response model, providing data
    only under request by a client:
    \begin{itemize}
      \item service server: who provides the service (response)
      \item service client: who call the service (request)
    \end{itemize}


  \section*{Parameters}
    Parameters are configuration values of a node, corresponding
    to passing arguments to an executable.


  \section*{Actions}
    Actions are an advance way to exchange data for long running
    tasks, using the client-server model of the topics and 
    the call-and-response model of services, with the difference 
    that can be deleted.
    \begin{itemize}
      \item goal service: service request
      \item feedback topic: data about the state of the task
      \item result service: service result
    \end{itemize}





\part{Client Libraries}



  \begin{tabularx}{\linewidth}{| l | l | l |}
    / & rclcpp & rclpy \\
    / & rclcpp::ok() & / \\
    / & rclcpp::WallRate & / \\
  \end{tabularx}




  