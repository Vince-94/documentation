\part{Filesystem}


%%%%%%%%%%%%%%%%%%%%%%%%%%%%%%%%%%%%%%%%%%%%%%%%%%%%%%%%%%
%%%%%%%%%%%%%%%%%%%%%%%% Package %%%%%%%%%%%%%%%%%%%%%%%%%
%%%%%%%%%%%%%%%%%%%%%%%%%%%%%%%%%%%%%%%%%%%%%%%%%%%%%%%%%%

\section{Package}

  The package is the smallest unit that is possible to build in ROS and the main unit for organizing software, which can contain:
  \begin{itemize}
      \item ROS runtime processes (nodes)
      \begin{itemize}
          \item scripts: contains executables python scripts
          \item src: contains C++ cource code
          \item include: contains headers and libraries needed
          \item launch: contains launch files
      \end{itemize}    
      \item ROS-dependent libraries (dependencies)
      \begin{itemize}
          \item msg: custom messages structure definition
          \item src: custom services structure definition
          \item action: custom actions structure definition
          \item config: configuration files
      \end{itemize}    
      \item Configuration files
      \begin{itemize}
          \item package.xml: manifest file
          \item CMakeLists.txt: CMake build file
      \end{itemize}
      \item Anything else that is usefully organized together
  \end{itemize}    

\begin{minted}[bgcolor=LightGray, fontsize=\footnotesize, baselinestretch=1.2]{shell}
catkin_ws
  /src
    /pkg_name
      /config
        file.config
      /include
        /pkg_name
          header.h
      /scripts
        node.py
      /src
        node.cpp
      /launch
        node.launch
      /msg
        message.msg
      /srv
        message.srv
      /param
        message.param
      /action
        message.action
      package.xml
      CMakeLists.txt
\end{minted}




    
\subsection{package.xml}
    The package manifest provides metadata about a package, including its name, version, description, license information, dependencies, and other meta information like exported packages.

\begin{minted}[bgcolor=LightGray, fontsize=\footnotesize, baselinestretch=1.2]{xml}
<package format="2">

  <name>pkg_name</name>
  <version>1.2.4</version>
  <description>pkg_description</description>
  <maintainer email="maintainer_email">maintainer_name</maintainer>
  <license>[sw license]</license>
  <url>http://ros.org/wiki/pkg_name</url>
  <author>author_name</author>

  <buildtool_depend>catkin</buildtool_depend>

  <! dependency only for building >
  <build_depend>...</build_depend>

  <! dependency only for building which exports a header >
  <build_export_depend>...</build_export_depend>

  <! dependency required only when running the package >
  <exec_depend>...</exec_depend>

  <! dependency including build_depend, build_export_depend, exec_depend >
  <depend>...</depend>

</package>
\end{minted}
    





\subsection{CMakeLists.txt}
    The CmakeLists file describes the build instructions for the Cmake.

\begin{minted}[bgcolor=LightGray, fontsize=\footnotesize, baselinestretch=1.2]{xml}
########## Informations ##########
cmake_minimum_required()
project(pkg_name)

# Compiler
set(CMAKE_BUILD_TYPE "Release")
include(CheckCXXCompilerFlag)
CHECK_CXX_COMPILER_FLAG("-std=c++11" COMPILER_SUPPORTS_CXX11)
CHECK_CXX_COMPILER_FLAG("-std=c++14" COMPILER_SUPPORTS_CXX14)

if(COMPILER_SUPPORTS_CXX14)
    set(CMAKE_CXX_FLAGS "-std=c++14")
elseif(COMPILER_SUPPORTS_CXX11)
    set(CMAKE_CXX_FLAGS "-std=c++11")
else()
    message(FATAL_ERROR "No C++11 support")
endif()

set(CMAKE_CXX_FLAGS_RELEASE "-O3 -Wall -g")


########## Dependencies ##########
# Find ROS package dependency (<depend>/<build_depend> in package.xml)
find_package(catkin REQUIRED COMPONENTS 
  rospy
  roscpp
  ...
)


########## Catkin specific configuration ##########
# Libraries and headers needed by all the interfaces you export to other 
# ROS packages (<depend>/<build_export_depend> in package.xml)
catkin_package(
  DEPENDS rospy roscpp ...
  CATKIN_DEPENDS rospy roscpp ...
)

# Header directories
include_directories(
  include 
  ${catkin_INCLUDE_DIRS}
)

\end{minted}




    %%%%%%%%%%%%%%%%%%%%%%%%%%%%%%%%%%%%%%%%%%%%%%%%%%%%%%%%%%
    %%%%%%%%%%%%%%%%%%%%% Data Structure %%%%%%%%%%%%%%%%%%%%%
    %%%%%%%%%%%%%%%%%%%%%%%%%%%%%%%%%%%%%%%%%%%%%%%%%%%%%%%%%%

\section{Data Structure}

    The message data structure is the specific format used to compose messages:
    \begin{itemize}
        \item .msg – topic data structure
        \item .srv – service data structure
        \item .action – action data structure
    \end{itemize}
    
    
    \subsection{Creating message package}
    
        It is always suggested to create a separate package containing all the custom messages created:

\begin{minted}[bgcolor=LightGray, fontsize=\footnotesize, baselinestretch=1.2]{xml}
catkin create pkg msg_dep -c rospy roscpp message_generation actionlib actionlib_msgs ...
\end{minted}
        with the following architecture:
\begin{minted}[bgcolor=LightGray, fontsize=\footnotesize, baselinestretch=1.2]{shell}
catkin_ws
  /src
    /pkg_name
      /msg
        message.msg
      /srv
        message.srv
      /param
        message.param
      /action
        message.action
      package.xml
      CMakeLists.txt
\end{minted}
        
        
        \subsection{msgs data structure}
\begin{minted}[bgcolor=LightGray, fontsize=\footnotesize, baselinestretch=1.2]{python}
# defining topic message
message_type message_name
\end{minted}
        
        \subsection{srv data structure}
\begin{minted}[bgcolor=LightGray, fontsize=\footnotesize, baselinestretch=1.2]{python}
# defining request message
request_type request_name
---
# defining response message
response_type response_name
\end{minted}

        \subsection{action data structure}
\begin{minted}[bgcolor=LightGray, fontsize=\footnotesize, baselinestretch=1.2]{python}
# defining topic message
# goal definition
goal_type goal_name
---
# result definition
result_type result_name
---
# feedback
feedback_type feedback_name
\end{minted}
        

        \subsection{CMakeLists.txt}
\begin{minted}[bgcolor=LightGray, fontsize=\footnotesize, baselinestretch=1.2]{shell}
find_package(catkin REQUIRED COMPONENTS
  rospy
  roscpp
  message_generation
  actionlib_msgs
)

# Messages data structure
add_message_files(
  DIRECTORY msg
  FILES message.msg
)

# Services data structure
add_service_files(
  DIRECTORY srv
  FILES message.srv
)

# Actions data structure
add_action_files(
  DIRECTORY action 
  FILES massage.action
)

generate_messages(
  DEPENDENCIES ...
  actionlib_msgs
)

catkin_package(
  CATKIN_DEPENDS rospy roscpp message_runtime actionlib_msgs
)
\end{minted}


        \subsection{package.xml}
\begin{minted}[bgcolor=LightGray, fontsize=\footnotesize, baselinestretch=1.2]{xml}
<build_depend>message_generation</build_depend>
<build_export_depend>message_runtime</build_export_depend>
<exec_depend>message_runtime</exec_depend>
<depend>actionlib</depend>
<depend>actionlib_msgs</depend>
\end{minted}





%%%%%%%%%%%%%%%%%%%%%%%%%%%%%%%%%%%%%%%%%%%%%%%%%%%%%%%%%%
%%%%%%%%%%%%%%%%%%%%%% Dependencies %%%%%%%%%%%%%%%%%%%%%%
%%%%%%%%%%%%%%%%%%%%%%%%%%%%%%%%%%%%%%%%%%%%%%%%%%%%%%%%%%



\section{Dependencies}
    

  \subsection{ROS packages}


  
  \subsection{System dependencies}
  Python modules and C++ libraries are dependencies that can be imported in any other package in the workspace.


  For python modules:
  \begin{itemize}
    \item package.xml
    \begin{itemize}
      \item <exec\_dep>: system dependencies (rosdep, rospkg, ecc...)
    \end{itemize}
  \end{itemize}
  
  For C++ libraries:
  \begin{itemize}
    \item package.xml
    \begin{itemize}
      \item <build\_dep>: packages needed for building your programs, including development files like headers, libraries and configuration files. For each build dependency, specify the corresponding rosdep key
      \item <build\_export\_depend>: packages needed for building your programs and furthermore that exports a header including an imported dependency
      \item <exec\_dep>: shared (dynamic) libraries, executables, Python modules, launch scripts and other files required for running your package
    \end{itemize}
    \item CMakeLists.txt
    \begin{itemize}
      \item find\_package: find imported libraries
      \item include\_directories: include libraries directories
    \end{itemize}
  \end{itemize}


  