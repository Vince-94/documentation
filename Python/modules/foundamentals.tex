\subsection{Data types}

\begin{tabularx}{\linewidth}{@{}| l | l | X |@{}}
    \hline
    Name & Data type & Description \\
    \hline
    string & 'str' & sequence of characters \\ 
    integer & 1 & integers numbers \\
    float & 0.1 & decimals numbers \\
    complex & 1j & imaginary numbers \\
    boolean & True/False & logical values \\
    list & [a, b, ...] & mutable ordered sequence \\
    tuple & (a, b, ...) & immutable ordered sequence \\
    range & range(init, fin, step) & immutable ordered sequence \\
    dictionary & \{key=value, ...\} & mutable unordered mapping \\
    set & \{x, ...\} &  mutable unordered set \\
    frozenset & frozenset(\{x, ...\}) & immutable unordered set \\
    byte & & \\
    bytearray & & \\
    memoryview & & \\
    \hline
\end{tabularx}



\subsubsection{Data conversion}

\begin{tabular}{@{}ll@{}}
    \verb!!    & convert in string data type \\
    \verb!int()!    & convert in integer data type \\
    \verb!float()!    & convert in float data type \\
    \verb!bool()!    & convert in bool data type \\
    \verb!list()!    & convert in list data type \\
    \verb!dict()!    & convert in dict data type \\
    \verb!set()!    & convert in set data type \\
\end{tabular}



\subsubsection{Generic operations}
\begin{tabular}{@{}ll@{}}
    \verb!len(x)!    & variable length \\
    \verb!min(x), max(x)!    & min/max value \\
    \verb!sorted(x)!    & sort a list \\
    \verb!enumerate(x, ...)!    & iterator \\
    \verb!zip(x)!    & iterator on tuple \\
    \verb!all(x)!    & true if all are true \\
    \verb!any(x)!    & true if at least one is true \\
\end{tabular}


\subsubsection{Sting operations}
\begin{tabular}{@{}ll@{}}
    \verb!s1 + s2!    & concatenation \\
    \verb!s * int!    & multiplication \\
    \verb!s[i]!    & indexing \\
    \verb!s.index()!    & ?? \\
    \verb!s.find()!    & ?? \\
    \verb!s.strip()!    & remove white spaces \\
    \verb!s.lower()!    & lowercase \\
    \verb!s.upper()!    & uppercase \\
    \verb!s.split([sep])!    & creates string array of words \\
    \verb!s.join(seq)!    & creates string array of words \\
    \verb!s.find('string')!    & match string in string \\
    \verb!s.replace('old', 'new')!    & replace string \\
\end{tabular}


\subsubsection{List operations}
\begin{tabular}{@{}ll@{}}
    \verb!l.append(x)!    & append a element \\
    \verb!l.extend(x)!    & append as sequence of elements \\
    \verb!l.insert(i,x)!    & insert variable at i position \\
    \verb!l.remove(x)!    & remove value x \\
    \verb!l.pop([i])!    & remove and return item at index i \\
    \verb!l.sort()!    & sort \\
    \verb!l.revert()!    & reverse sort \\
\end{tabular}


\subsubsection{Dictionaries operations}
\begin{tabular}{@{}ll@{}}
    \verb!d[key] = value!    & assign value to a key \\
    \verb!d.clear()!    & clear dictionary \\
    \verb!d.update(d2)!    & ?? \\
    \verb!d.key()!    & ?? \\
    \verb!d.values()!    & ?? \\
    \verb!d.items()!    & ?? \\
    \verb!d.pop(key[default])!    & ?? \\
    \verb!d.popitem()!    & ?? \\
    \verb!d.get(key[default])!    & ?? \\
    \verb!d.setdefault(key[default])!    & ?? \\
\end{tabular}


\subsubsection{Sets operations}
\begin{tabular}{@{}ll@{}}
    \verb!s.update(s2)!    & ?? \\
    \verb!s.copy()!    & ?? \\
    \verb!s.add(key)!    & ?? \\
    \verb!s.remove(key)!    & ?? \\
    \verb!s.discard(key)!    & ?? \\
    \verb!s.clear()!    & ?? \\
    \verb!s.pop()!    & ?? \\
\end{tabular}






\section{Operators}

    \subsection{Arithmetic operators}
        \begin{tabular}{@{}ll@{}}
            \verb!+, -, *, /!    & basic operations \\
            \verb!%!     & module \\
            \verb!**!      & exponential \\
            \verb!//!      & floor \\
        \end{tabular}
        
    \subsection{Logic operators}
        \begin{tabular}{@{}ll@{}}
            \verb!==!    & equal \\
            !=    & not equal \\
            \verb!<,>!     & strict inequality \\
            \verb!<=,>=!      & inequality \\
        \end{tabular}


    \subsection{Boolean operators}
        \begin{tabular}{@{}ll@{}}
            \verb!x and y!    & and operator \\
            \verb!x or y!    & or operator \\
            \verb!not x!    & not operator \\
        \end{tabular}
        
        
    \subsection{Assignment operators}
        \begin{tabular}{@{}ll@{}}
            \verb!=!    & value assignment \\
            \verb!+=, -=, *=, /=, **=, //=!    & arithmetic assignment \\
            % \verb!&=, \|=!    & logic assignment \\
            % \verb!^=!    & ?? \\
            % \verb!>>=, <<=!    & ?? \\
        \end{tabular}
        
        
    \subsection{Other operators}
        \begin{tabular}{@{}ll@{}}
            \verb!x is y!    & true if x is y \\
            \verb!x is not y!    & true if x is not y \\
            \verb!x in y!    & true x is in y sequence \\
            \verb!x not in y!    & true x is not in y sequence \\
        \end{tabular}
        
        
    \subsection{Indexing}
        \begin{tabular}{@{}ll@{}}
            \verb!x[i]!    & list i-th element \\
            \verb!x[init_val:fin_val]!    & list slicing \\
        \end{tabular}







\subsection{Loops/Conditions}

\subsubsection{if/elif/else}

\begin{minted}{python}
if condition:
    ...
elif condition:
    ...
else:
    default
\end{minted}


\subsubsection{while}

\begin{minted}{python}
while condition:
    ...
\end{minted}


\subsubsection{for}

\begin{minted}{python}
\for condition:
    ...
\end{minted}


Keywords:
\begin{tabular}{@{}ll@{}}
    \verb!break!    & end the loop \\
    \verb!continue!    & continue the loop \\
    \verb!\t!    & tab \\
\end{tabular}


    


\subsection{Functions}

\begin{minted}{python}
# function definition
def fun_name(param, *args, param=default, **kwargs):
    # function body
    ...
    return x    # value to return

# function call
a = fun_name(arg_1, ..., arg_n))
\end{minted}





\subsection{Input/Output}

\begin{tabular}{@{}ll@{}}
    \verb!print(x)!    & print function \\
    \verb!x = input("string: ")!    & input function \\
\end{tabular}

Keywords:
\begin{tabular}{@{}ll@{}}
    \verb!\n!    & newline \\
    \verb!\s!    & space \\
    \verb!\t!    & tab \\
\end{tabular}










\subsection{Filesystem}

\begin{tabular}{@{}ll@{}}
    \verb!f = open("file","mode",encoding="enc")!    & open file \\
    \verb!f.write(text)!    & writing \\
    \verb!f.writelines(list of lines)!    & writing \\
    \verb!f.read()!    & reading chars \\
    \verb!f.readlines()!    & reading lines \\
    \verb!f.readline()!    & next line \\
    \verb!f.close()!    & close file \\
\end{tabular}\\
mode: \\
- r: read \\
- a: append \\
- w: write \\
- x: create \\
enc: \\
- utf8 \\
- ascii \\
- latin1 \\


\begin{minted}{python}
# open file
with open("file", "mode") as f:
    for line in f:
        ...
\end{minted}







\subsection{Error Handling}

\begin{minted}{python}
raise ExcClass()
try:
    <code>
except Exception as exc:
    <error>
\end{minted}




