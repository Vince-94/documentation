\subsection{Data types}

\begin{tabularx}{\linewidth}{@{}| l | l | X |@{}}
    \hline
    Name & Data type & Description \\
    \hline
    string & 'str' & sequence of characters \\ 
    integer & 1 & integers numbers \\
    float & 0.1 & decimals numbers \\
    complex & 1j & imaginary numbers \\
    boolean & True/False & logical values \\
    list & [a, b, ...] & mutable ordered sequence \\
    tuple & (a, b, ...) & immutable ordered sequence \\
    range & range(init, fin, step) & immutable ordered sequence \\
    dictionary & \{key=value, ...\} & mutable unordered mapping \\
    set & \{x, ...\} &  mutable unordered set \\
    frozenset & frozenset(\{x, ...\}) & immutable unordered set \\
    byte & & \\
    bytearray & & \\
    memoryview & & \\
    \hline
\end{tabularx}




\subsection{Operators}




\subsection{Loops/Conditions}





\subsection{Functions}

\begin{minted}{python}
# function definition
def fun_name(param, *args, param=default, **kwargs):
    # function body
    ...
    return x    # value to return

# function call
a = fun_name(arg_1, ..., arg_n))
\end{minted}





\subsection{Input/Output}




\subsection{Filesystem}




\subsection{Error Handling}




