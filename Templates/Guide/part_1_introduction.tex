\part{Introduction}

    \section{Definition}
        ROS is an open-source, meta-operating system for robotic systems, providing:
        \begin{itemize}
            \item hardware abstraction
            \item low-level device control
            \item implementation of commonly used functionality
            \item message-passing between processes
            \item package management
            \item tools and libraries for obtaining, building, writing, and running code across multiple computers
        \end{itemize}
        The purpose of ROS is:
        \begin{itemize}
            \item Support code reusability in robotics research and development
            \item Thin: code written for ROS can be used with other robot software frameworks
            \item ROS-agnostic libraries: the preferred development model is to write ROS-agnostic libraries with clean functional interfaces
            \item Language independence: implementation in different programming language (most used C++ and Phyton)
            \item Easy testing: built-in unit/integration test framework called rostest
            \item Scaling: ROS is appropriate for large runtime systems and for large development processes.
        \end{itemize}


    \section{Description} 
        ROS is a distributed framework of processes called nodes which enable executables to be individually designed and loosely coupled at runtime. 
        Communication between nodes is implemented by means of topics, services, actions and parameters, for each of which is defined a data structure. 
        Messages can be recorded through bag functionality which allows to reproduce offline them. The launch functionality allows to run multiple nodes 
        simultaneously and settings their parameters. This represents the \textbf{\emph{computational graph level}}. \\
        \\
        Nodes and in general anything that is useful to organize together, are organized into packages, which are the smallest unit that is possible to 
        build in ROS, fulfilling a defined function and defined by their own configurational files. This is the base of the \textbf{\emph{filesystem level}}. 
        Packages can be grouped into stacks, which can be built as packages in a nested framework. Furthermore, within packages can be defined message structures. \\
        \\
        ROS features are accessible via code through \textbf{\emph{client libraries}}, allowing to implement nodes in different coding languages. Actually, 
        the most supported are C++ (roscpp) and Python (rospy), alongside LISP (roslisp) and Matlab/Simulink. \\
        \\
        The \textbf{\emph{community level}} is the essence of ROS. Packages and stacks can be shared across the ROS community through a federated system 
        of online repositories. \\
        \\
        Furthermore, a \textbf{\emph{higher-level architecture}} is necessary to build larger systems on top of ROS.


            